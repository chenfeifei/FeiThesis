\startchapter{Simplified Molecular Model} \label{ch:3}

\section{Description}
The goal of Chapter \ref{ch:3} is to further expand this research focus and introduce the formulas used to generate our LP models. Before studying real life molecules, a toy molecule with limited vibration modes is first studied. By doing so, the nature of the LP formulas used to gain some further insights of the models we need is carefully analyzed. Our goal is to figure out with the all the spectral information available, could LP models we build output any valuable information. \\

A toy molecule with $4$ vibration modes is constructed. Theses vibrational peaks are at frequencies of $2850, 2960, 3050$ and $3200$. The widths of the peaks are $5, 10, 5$ and $15~cm^{-1}$, respectively. The amplitude of the peak are $1, 0.7, -0.2$ and $0.5~cm^{-1}$, respectively. The comparing angles of the peaks are $15, 90, 0$ and $60$. (TODO: check with Dennis, how to explain those comparing angles?)  \\

For this toy model, only IR spectroscopy is considered. Equation \ref{eq:3.1} is used to generate the cosine projection IR spectroscopy. Moreover, both $\phi$ and $\psi$ Euler angles are integrated, only the difference on angle $\theta$ is considered for the toy model as well. \\

\begin{eqnarray} \label{eq:3.1}
& f_{\theta}(x) = \displaystyle\sum^{4}_{q=1} A_q^2 * cos^2(\theta - \theta_q)\frac{\gamma^2}{(x- \omega_{\rm q})^2 + \gamma^2} 
\end{eqnarray}
where $A$ is the amplitude, $\theta_{q}$ is the $tilt$ angle of the candidate, $\gamma$ is the width, and $\omega_{\rm q}$ is the frequency. (TODO: Double check the correct meaning of each symbol) Ten different candidates with $10$ different $\theta$ values as follows: $0^{\circ}$, $10^{\circ}$, $20^{\circ}$, $30^{\circ}$, $40^{\circ}$, $50^{\circ}$, $60^{\circ}$, $70^{\circ}$, $80^{\circ}$, $90^{\circ}$. Their spectra are generated as shown in Figure \ref{fig:3.1}. The 10 candidates have peaks at the same frequencies. The spectral signal for candidates is comparatively strong at each peak. \\

\begin{figure}[!ht] \label{fig:3.1}
\centering
\includegraphics[scale=0.4]{Figures/Toy_Model_IR_Cosine_Projection.png} 
\caption{Toy model IR candidates cosine polarization} 
\end{figure}

%A target spectrum is composed by combining $15$ percent of candidate with $\theta$ of $20^{\circ}$ and $85$ percent of candidate of $\theta$ of $70^{\circ}$: $0.15*f_{20}(x) + 0.85*f_{70}(x)$ in the following experiment. \\

\section{Linear Programming Model for Spectra Study}

The LP model constructed to check if the optimal solution returned by the LP solver actually matches the target composition, is shown in Equation \ref{eq:3.2}. This model has also been used to study the composition of Ribonucleic acid (RNA) with ultraviolet (UV) spectra \cite{NYAS:NYAS900} and other UV spectroscopy studies  \cite{LPATUAS} back in the 60s. \\

\begin{eqnarray} \label{eq:3.2}
& minimize \displaystyle\sum^{number of points}_{p=1} \left| Target- \displaystyle\sum^{number of candidates}_{c=1}p_{c}f_{\theta}(x) \right| 
\end{eqnarray}
where $p_{c}$ are the unknown percentages for each candidate, which are the decision variables. $p$ is the number of points selected along the wavenumber, both for candidates and target spectra. Target refers to the corresponding data points selected in target spectra. For each data point, the absolute residual between target spectrum and the one composed by the decision variables is calculated. The objective function minimizes the sum of the absolute residuals over all the data points. \\

However, in order to use a LP approach, getting rid of the absolute signs in the objective function is needed. Because Equation \ref{eq:3.2} is subject to no restrictions, meanwhile, the objection function is not in standard form. To eliminate the absolute sign is achieved by introducing one more variable $X$ and two more constraints for each data point as shown in Equation \ref{eq:3.3}. Then the previous model in Equation \ref{eq:3.2} is converted into the one in Equation \ref{eq:3.4}, which can be solved by an LP solver. At last, one more constraint is introduced to restrict the sum of the percentages to be 1, as shown in Equation \ref{eq:3.4}. \\

For each point in the range of valid wavenumbers:
\begin{eqnarray} \label{eq:3.3}
& X = \left| Target-\displaystyle\sum^{candidates}_{c=1}p_{c}f_{\theta}(x) \right| \nonumber \\
&  X \geq Target-\displaystyle\sum^{candidates}_{c=1}p_{c}f_{\theta}(x)   \nonumber \\
& X \geq -Target+\displaystyle\sum^{candidates}_{c=1}p_{c}f_{\theta}(x)  
\end{eqnarray} 

\begin{eqnarray} \label{eq:3.4}
& minimize \displaystyle\sum^{points}_{p=1} X_p \nonumber \\
& X_1 - Target_1 + \displaystyle\sum^{candidates}_{c=1}p_{c}f_{\theta}(x_1) \geq 0 \nonumber \\
& X_1 + Target_1 - \displaystyle\sum^{candidates}_{c=1}p_{c}f_{\theta}(x_1) \geq 0 \nonumber \\
& ... \nonumber \\
& X_n - Target_n + \displaystyle\sum^{candidates}_{c=1}p_{c}f_{\theta}(x_n) \geq 0 \nonumber \\
& X_n + Target_n - \displaystyle\sum^{candidates}_{c=1}p_{c}f_{\theta}(x_n) \geq 0 \nonumber \\
& \displaystyle\sum^{candidates}_{c=1}p_{c} = 1 
\end{eqnarray} 

\section{Linear programming model implementation}
To start, code is written to generate a file that contains all the candidates' spectral information needed for the experiments. For this step, the molecular properties files are used. For a specific candidate, given a molecular properties file and a $\theta$ value, the candidate's spectral information is obtained. For toy model, only the value of $\theta$ is needed, then Equation \ref{eq:3.1} is used to synthesize the spectral information. \\

In the first step, a candidate class is written. This class defines candidate's $x$- and $z$- polarized IR spectra; $xx$-, $xy$-, $xz$-, and $zz$- polarized Raman spectra; $yyz$-, $yzy$-, $zzz$- polarized SFG spectra. Given candidate's molecular properties and $\theta$ value, a instance of this specific candidate is created. For the toy model, it only contains IR spectral information. Therefore, one candidate only contains $cosine$- and $sine$- polarized IR spectra. \\

In the second step, more code is written to generate a target composition of a list of needed candidates. Then the target composition is used to generate the target spectra. The probe range, which is the range of the wavenumber, is from 2800 to 3300. For further experiments in Chapter \ref{ch:4}, \ref{ch:5} and \ref{ch:6}, the probe arrange is from $2000$ wavenumber to $3000$ wavenumber. For toy model, the probe range is from $2800$ wavenumber to $3800$ wavenumber. Then all the spectral information of candidates and target is created in a text file. The code can also be used to run experiments that contain different spectroscopy information. For example, one file can contain only candidates and target's IR spectral information, or contain all three spectroscopy information. \\

In the third step, the LP model is constructed by using the spectral information text file. This part of the code was written by Hung \cite{KuoKaiHung:Thesis:2015}. It reads all the candidates and target spectral information, and builds the LP model as shown in Equation \ref{eq:3.4}, then creates CPLEX LP input file. \\

In the fourth step, we use LP solver ``GNU linear progarmming tool kit" (GLPK) to obtain the result. \\
\section{Experiments}
In the first experiment set, $4$ candidates are selected as shown in Table \ref{tab:3.1}. The table illustrates the detailed settings for Experiment 1 and 2. In Experiment 1, there are $4$ candidates with $\theta$ of $0^{\circ}, 10^{\circ}, 20^{\circ}$, and $30^{\circ}$. For Experiment 2, the $\theta$ values are changed to $0^{\circ}, 5^{\circ}, 10^{\circ}$, and $15^{\circ}$. Instead of having a $10$ degree variance in $\theta$, $5$ degree difference is applied on $\theta$ for Experiment 2. This means that when the candidates become more similar to each other as their spectra are more similar. In both experiments, 100 data points are selected evenly along the wavenumber from the spectra of $cosine$-polarized IR. The target composition of the candidates are the same for both experiments. In Experiment 1, the return composition is the same as the target one, however, the return composition for Experiment 2 does not match to the target one. %The increase in similarity results in the problem more difficult to resolve. 

\begin{table} \label{tab:3.1}
\begin{center}
\begin{tabular}{| l | l | l | }
\hline
Experiment index & 1 & 2  \\
\hline
Number of Candidates & 4 & 4  \\
\hline
Candidates & [0, 10, 20, 30] & [0, 5, 10, 15] \\
\hline
Target Composition & [0.1, 0.5, 0.4, 0] & [0.1, 0.5, 0.4, 0]     \\
\hline
Number of Data Points & 100(cos) &  100(cos)     \\
\hline
Return Composition & [0.1, 0.5, 0.4, 0] & [0, 0.796962, 0.103038, 0.1] \\
\hline
\end{tabular} 
\end{center}
\caption{Experiment 1 and 2 Setting}
\end{table}	

In order to figure out why the return composition in Experiment 2 is different from the target one, the spectra generated by the return composition is plotted together with the target spectra as shown in Figure \ref{fig:3.2}. Note that the result spectra is almost identical to the target one. The residual between them is almost $0$. In order to see whether this observation is a general case, another experiment Experiment 3 is set up in Table \ref{tab:3.2}. Experiment 3 contains more candidates than Experiments 1 and 2. $10$ candidates are included with $\theta$ values ranging from $0^{\circ}$ to $90^{\circ}$.  \\

\begin{figure}[!ht] \label{fig:3.2}
\centering
\includegraphics[scale=0.7]{Figures/toy_model_result_plotting_ir_cos_4candi_1.png}
\caption{Toy Model Result Plotting for 4 Candidates on IR Cosine Projection}
\end{figure}

\begin{table} \label{tab:3.2}
\begin{center}
\begin{tabular}{| l | p{7cm} | }
\hline
Experiment index & 3  \\
\hline
Number of Candidates & 10   \\
\hline
Candidates & [0, 10, 20, 30, 40, 50, 60, 70, 80, 90]  \\
\hline
Target Composition & [0.1, 0, 0.5, 0, 0.4, 0, 0, 0, 0, 0] \\
\hline
Number of Data Points & 100(cos) \\
\hline
Return Composition & [0, 0, 0.730541, 0, 0.212061,0, 0, 0.0573978, 0, 0] \\
\hline
\end{tabular}
\end{center}
\caption{Experiment 3 Setting}
\end{table}	

\begin{figure}[!ht] \label{fig:3.3}
\centering
\includegraphics[scale=0.7]{Figures/toy_model_result_plotting_ir_cos_10candi_1.png} 
\caption{Toy Model Result Plotting for 10 Candidates on IR Cosine Projection}
\end{figure}

Table \ref{tab:3.2} indicates the return composition for Experiment 3 is different from the target one. Figure \ref{fig:3.3} shows that the spectrum produced by the return composition is almost identical to the one generated by the target composition. The residual is negligible. The result is the same as Experiment 2. \\

Among Experiment 1, 2 and 3, only Experiment 1 return composition matches its target one. For Experiment 2 and 3, the return composition is totally different from the target one. However, for Experiment 2, the difference in $\theta$ among the candidates is smaller than Experiment 1. For Experiment 3, the number of the candidates is larger than Experiment 1. Both effects increase the complexity of the experiments. Therefore, further increase the difficulty for LP model to return the target composition. for both Experiment 2 and 3, the spectra constructed by the return composition matches to the one built by the target composition. \\

This demonstrates that there are multiple compositions can achieve to construct the spectra that are close to the target ones. However,the numerical limitation in the LP model helps the LP solver to converge to a unique one. Moreover, the reason for the LP model in Experiment 1 to return a composition that matches to the target one, is that the spectral information used to construct the LP model is competent. The constraints constructed in the LP model eventually converge to the target one, which results in better numerical outcome. The LP model is more complete compared to the one in Experiment 2. 

%the corresponding LP model, built with the presenting spectral information, may lack of competent information to guarantee the return composition actually  is the target one. Because of the high complexity in the setting of the candidates pool, the solution for the LP model may not be unique. However, in real LP problem solving, the numerical limitation helps LP solver to converge to an unique solution. This unique solution is the optimum one for the LP model with the information currently available. Further demonstration will also be expanded in Chapter \ref{ch:4} when real molecules are introduced. \\ %the complexity of Experiment 2 and 3 is higher than Experiment 1.\\

In order to add the necessary information to construct the constraints in our LP model, IR's second polarization introduced for the toy model: the sine polarization. Figure \ref{fig:3.4} describes how the spectra are presented for $10$ candidates same as Experiment 3. Experiment 4 and 5 will include both polarizations' spectral information when build the LP model. Table \ref{tab:3.3} displays the setting for Experiment 4 and 5. This setting is based on Experiment 2, with sine-polarization IR spectrum added. $100$ data points are selected from this additional spectra, then converted to additional decision variables and constraints in the LP model. Same with Experiment 5, it is based on Experiment 3, with sine-polarization IR spectral information added. For both Experiment 4 and 5, the return composition now matches to the target one. This further proves that as long as we have sufficing information to build the constraints, the LP solver will return a composition matches to the target one. \\ 

\begin{figure}[!ht] \label{fig:3.4}
\centering
\includegraphics[scale=0.7]{Figures/Toy_Model_IR_Sine_Projection.png} 
\caption{Toy Model Candidates IR Sine Projection} 
\end{figure}

\begin{table} \small \label{tab:3.3}
\begin{center}
\begin{tabular}{| l | p{5cm} | l |}
\hline
Experiment index & 4 & 5\\
\hline
Number of Candidates & 4 & 10 \\
\hline
Candidates & [0, 5, 10, 15] & [0, 10, 20, 30, 40, 50, 60, 70, 80, 90] \\
\hline
Target Composition & [0.1, 0.5, 0.4, 0] & [0.1, 0, 0.5, 0, 0.4, 0, 0, 0, 0, 0]\\
\hline
Number of Data Points & 100(cos) + 100(sin) & 100(cos) + 100(sin)\\
\hline
Return Composition & [0.1, 0.5, 0.4, 0] & [0.1, 0, 0.5, 0, 0.4, 0, 0, 0, 0, 0] \\
\hline
\end{tabular} 
\caption{Experiment 4 and 5 Setting}
\end{center}
\end{table}		

\section{Constraint Study Based on Experiment 4}

From Experiment 1 to 5, we know having sufficient information in our LP model is the key to obtain the target composition. Having sufficient information means having enough constraints to help LP model converge to a desired result. Moreover, the information is coming from the valuable data points selected along the spectra. This leads us to do a further study on the constraints in order to see how many data points are enough to get the desired composition.\\ 

Based on Experiment 4, experiments about formulating the LP model with different data information are conducted in Table \ref{tab:3.4}. The number of data points indicates how many data points are selected. Points Selection shows how data points are selected. [2800, 3300, 50] means along wavenumber from 2500 to 3300, every 25 wavenumber. For example, Experiment 6 contains 10 data points from cosine-polarized IR spectrum. Every 50 wavenumber, one data point is selected. Similarly, for Experiment 7, 8, 9, 10, 11, every 25, 20, 15, 10 and 5 wavenumber, one data point is select. From Experiment 12 to 16, data points are selected from both cosine-polarized and sine-polarized IR spectrum. \\

(TODO: rethink: What can we exactly get from the following two tables? Should we include this study?)

\begin{table} \small \label{tab:3.4}
\begin{center}
\begin{tabular}{| l | l | p{3cm} | l |} \hline
	Experiment Index & Number of Data Points & Points Selection & Result \\ \hline
	6 & 10 & [2800, 3300, 50] & [0, 0.796962, 0.103038, 0.1] \\ \hline
	7 & 20 & [2800, 3300, 25] & [0, 0.796962, 0.103038, 0.1] \\ \hline
	8 & 25 & [2800, 3300, 20] & [0, 0.796962, 0.103038, 0.1] \\ \hline
	9 & 32 & [2800, 3300, 15] & [0, 0.796962, 0.103038, 0.1] \\ \hline
	10 & 50 & [2800, 3300, 10] & [0, 0.796962, 0.103038, 0.1] \\ \hline
	11 & 100 & [2800, 3300, 5] & [0, 0.796962, 0.103038, 0.1] \\ \hline
	12 & $100 + 1$ & [2800, 3300, 5], [2800, 3300, 500] & [0, 0.796962, 0.103038, 0.1] \\ \hline
	13 & $100 + 5$ & [2800, 3300, 20], [2800, 3300, 100] & [0, 0.796962, 0.103038, 0.1] \\ \hline
	14 & $100 + 10$ & [2800, 3300, 20], [2800, 3300, 50] & [0, 0.796962, 0.103038, 0.1] \\ \hline
	15 & $100 + 50$ & [2800, 3300, 20], [2800, 3300, 10] & [0.1, 0.5, 0.4, 0] \\ \hline
	16 & $100 + 100$ & [2800, 3300, 20], [2800, 3300, 5] & [0.1, 0.5, 0.4, 0] \\ 
	\hline
\end{tabular} 
\end{center}
\caption{Constraint Study Based on Experiment 4}
\end{table}

One interesting result from Table \ref{tab:3.4} is that: from Experiment 1 to 9, the result composition is the same. To the contrary, from Experiment 10, the return composition gets changed to the target one. Furthermore, if we plot the return composition of [0, 0.796962, 0.103038, 0.1] and the target one [0.1, 0.5, 0.4, 0] in  Picture \ref{fig:3.5}. In this picture, we can see that the spectra generated by these two composition are identical.

\begin{figure}[!ht] \label{fig:3.5}
\centering
\includegraphics[scale=0.3]{Figures/toy_model_result_plotting_ir_sin_4candi_constraint_study_experiment4.png} 
\caption{Toy Model Constraint Study 1}
\end{figure}


\section{Constraint Study Based on Experiment 5}

When the same constraint study is applied to the data based on Experiment 5 in Table \ref{tab:3.6}, the observation is the same as the experiments in Table \ref{tab:3.4}. This further proves that: We can obtain different solutions by have different constraints. When the  result composition [0, 0, 0.730541, 0, 0.212061,0, 0, 0.0573978, 0, 0] and target one [0.1, 0, 0.5, 0, 0.4, 0, 0, 0, 0, 0] are plotted together, they are almost identical as well, as shown in Figure \ref{fig:3.6}.

\begin{table} \tiny \label{tab:3.6}
\begin{center}
\begin{tabular}{| l | l | p{3cm}  | p{6cm} |}
\hline
Experiment Index & Points & Point Selection & Result \\ \hline
17 & 10 & [2800, 3300, 50] & [0.156758, 0, 0, 0.825977, 0, 0, 0, 0, 0, 0.017265] \\ \hline
18 & 25 & [2800, 3300, 20] & [0, 0, 0.730541, 0, 0.212061, 0, 0, 0.0573978, 0, 0, 0] \\ \hline
19 & 50 & [2800, 3300, 10] & [0, 0, 0.730541, 0, 0.212061, 0, 0, 0.0573978, 0, 0, 0] \\ \hline
20 & 100 & [2800, 3300, 5] & [0, 0, 0.730541, 0, 0.212061, 0, 0, 0.0573978, 0, 0, 0] \\ \hline
21 & 500 & [2800, 3300, 5] & [0, 0, 0.730541, 0, 0.212061, 0, 0, 0.0573978, 0, 0, 0] \\ \hline	
22 & $100 + 1$ & [2800, 3300, 5], [2800, 3300, 500] & [0, 0, 0.730541, 0, 0.212061, 0, 0, 0.0573978, 0, 0, 0] \\ \hline
23 & $100 + 10$ & [2800, 3300, 5], [2800, 3300, 50] & [0.361587, 0, 0.312061, 0.326352, 0, 0, 0, 0, 0] \\ \hline
24 & $100 + 20$ & [2800, 3300, 5], [2800, 3300, 25] & [0.174023, 0, 0, 0.791447, 0, 0, 0.0345301, 0, 0, 0] \\ \hline
25 & $100 + 25$ & [2800, 3300, 20], [2800, 3300, 20] & [0.174023, 0, 0, 0.791447, 0, 0, 0.0345301, 0, 0, 0] \\ \hline
26 & $100 + 50$ & [2800, 3300, 5], [2800, 3300, 10] & [0, 0, 0.753209, 0, 0.146791, 0, 0.1, 0, 0, 0] \\ \hline
27 & $100 + 84$ & [2800, 3300, 5], [2800, 3300, 6] & [0.174023, 0, 0, 0.791447, 0, 0, 0.0345301, 0, 0, 0] \\ \hline
28 & $100 + 100$ & [2800, 3300, 5], [2800, 3300, 5] & [0.1, 0, 0.5, 0, 0.4, 0, 0, 0, 0, 0] \\ 
\hline
\end{tabular} \\
\caption{Constraint Study Based on Experiment 5}
\end{center}
\end{table}

\begin{figure}[!ht] \label{fig:3.6}
\centering
\includegraphics[scale=0.3]{Figures/toy_model_result_plotting_ir_sin_10candi_constraint_study_experiment5.png} 
\caption{Toy Model Constraint Study 2}
\end{figure}

\section{Discussion and Conclusion}
With all the experiments conducted with the toy model, we have learnt that the reason, that our LP model does not return a composition that matches the target one, is that the model does not have sufficient information to build the constraints. However, with the limited information, the optimal solution returned by our LP model does build the perfect target spectrum. This means that the solution for the composition that achieves minimum residual of the objective function is not unique. However, in real experiment, because numerical restriction, an unique optimal solution is obtained from the LP model. \\

Above analysis simulates the following question: how can we know there is enough information to achieve the target composition? In the next step, we will experiment with real molecules. The goal is to investigate with all the spectral information that we can obtain for real molecules, can our LP model return the target composition for the target spectrum? If yes, can we apply the LP model systematically? Furthermore, to maximally explore the capacity of our LP model, and study its limitation. Finally, come up with some general instructions for applying our LP model. These are the main focus for the following chapters.\\

		

		 



