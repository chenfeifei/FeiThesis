\startchapter{Simplified Molecular Model} \label{ch:3}

\section{Description}
The goal of this chapter is to introduce our LP model, as well as exploring its properties by using a simplified molecular model. The purpose of introducing the simplified molecular model is to limit the complexity that comes from the parameters needed to describe the realistic models, so that the analysis of the nature of the LP model can be focused. \\

Our simplified molecular model contains four vibration modes. Theses vibrational peaks are at frequencies of $2850$, $2960$, $3050$ and $3200~\rm cm^{-1}$. The widths of the peaks are $5$, $10$, $5$ and $15~\rm cm^{-1}$, respectively. The amplitudes of the peaks are $1$, $0.7$, $-0.2$ and $0.5~\rm cm^{-1}$, respectively. The comparing angles of the peaks are $15^{\circ}, 90^{\circ}, 0^{\circ}$ and $60^{\circ}$. \\

Only IR spectroscopy is considered for the simplified molecular model. Equation \ref{eq:3.1} is used to generate the $z$-polarized IR spectrum. Moreover, both Euler angles $\varphi$ and $\psi$ are integrated into uniform distribution, only the difference on angle $\theta$ is considered. \\

\begin{eqnarray} \label{eq:3.1}
& f_{\theta}(\omega_{\rm IR}) = \displaystyle\sum^{4}_{q=1} A_q^2 cos^2(\theta - \theta_q)\frac{\Gamma^2}{(\omega_{\rm IR}- \omega_{\rm q})^2 + \Gamma^2} 
\end{eqnarray}
where $A_{q}$ is the amplitude, $\theta_{q}$ is the comparing angle, $\Gamma$ is the width, and $\omega_{\rm q}$ is the frequency. Ten candidates are produced with ten different $\theta$ values as follows: $0^{\circ}$, $10^{\circ}$, $20^{\circ}$, $30^{\circ}$, $40^{\circ}$, $50^{\circ}$, $60^{\circ}$, $70^{\circ}$, $80^{\circ}$, and $90^{\circ}$. Their spectra are shown in Figure \ref{fig:3.1}. The 10 candidates have peaks at the same frequencies. \\

\begin{figure}[!ht] 
\centering
\includegraphics[scale=0.7]{Figures/Toy_Model_IR_Cosine_Projection.png} 
\caption{$z$-polarized IR spectra of candidates of simplified molecular model.} \label{fig:3.1}
\end{figure}

%A target spectrum is composed by combining $15$ percent of candidate with $\theta$ of $20^{\circ}$ and $85$ percent of candidate of $\theta$ of $70^{\circ}$: $0.15*f_{20}(x) + 0.85*f_{70}(x)$ in the following experiment. \\

\section{Linear Programming Model for Spectral Study}

Equation \ref{eq:3.2} describes the objective function that build the basis of our LP model, as well as one constraint that limits the sum of all the candidates' percentage to $100\%$. \\

\begin{eqnarray} \label{eq:3.2}
& \underset{p_{c}} {\text{minimize}} \displaystyle\sum^{N_{p}}_{n=1} \left| \text{Target}- \displaystyle\sum^{N_{c}}_{c=1}p_{c}f_{\theta}(x) \right|   \nonumber \\
& \displaystyle\sum^{N_{c}}_{c=1}p_{c} = 1 
\end{eqnarray}
where $p_{c}$ are the unknown percentages of the candidate, which are the decision variables. These percentages are returned by LP solver, and called return composition in future test cases. $N_{p}$ is the number of points selected along the wavenumber, both for candidates and target spectra. $\rm Target$ refers to the corresponding data points selected in target spectra. $N_{c}$ is the number of candidates. For each data point, the absolute residual between the target spectrum and the one composed by the decision variables is calculated. The objective function minimizes the sum of the absolute residuals over all the data points. \\

The optimal solution returned by the LP solver is then compared with the target composition to see if they match each other. This equation has also been used to study the composition of Ribonucleic acid (RNA) with ultraviolet (UV) spectra \cite{NYAS:NYAS900} and other UV spectroscopy studies  \cite{LPATUAS} back in the 60s. \\

However, because of the absolute signs in the objective function, Equation \ref{eq:3.2} is not an LP problem. We transform Equation \ref{eq:3.2} by getting rid of the absolute signs. We introduce one more variable $\rm X$ and two more constraints to each data point as shown in Equation \ref{eq:3.3}. The previous model in Equation \ref{eq:3.2} is then converted into the one in Equation \ref{eq:3.4}, and it can be solved by LP solvers. \\

\begin{eqnarray} \label{eq:3.3}
& X = \left| \rm Target-\displaystyle\sum^{N_{p}}_{c=1}p_{c}f_{\theta}(x) \right| \nonumber \\
&  X \geq \rm Target-\displaystyle\sum^{N_{c}}_{c=1}p_{c}f_{\theta}(x)   \nonumber \\
& X \geq -\rm Target+\displaystyle\sum^{N_{c}}_{c=1}p_{c}f_{\theta}(x)  
\end{eqnarray} 

\begin{eqnarray} \label{eq:3.4}
& minimize \displaystyle\sum^{N_{p}}_{n=1} X_p \nonumber \\
& \rm X_1 - \rm Target_1 + \displaystyle\sum^{N_{c}}_{c=1}p_{c}f_{\theta}(x_1) \geq 0 \nonumber \\
& \rm X_1 + \rm Target_1 - \displaystyle\sum^{N_{c}}_{c=1}p_{c}f_{\theta}(x_1) \geq 0 \nonumber \\
&\vdots  \nonumber \\
& \rm X_n - \rm Target_n + \displaystyle\sum^{N_{c}}_{c=1}p_{c}f_{\theta}(x_n) \geq 0 \nonumber \\
& \rm X_n + \rm Target_n - \displaystyle\sum^{N_{c}}_{c=1}p_{c}f_{\theta}(x_n) \geq 0 \nonumber \\
& \displaystyle\sum^{N_{c}}_{c=1}p_{c} = 1 
\end{eqnarray} 

Note that the LP model exactly describes our problem to be solved. Assuming that we can obtain sufficiently precise data, solving the LP will yield the target composition. Recall that if the solution space of an LP instance is feasible and bounded, then there is a unique optimum solution. \\

\section{Linear Programming Model Implementation}

Next, I describe how to solve instances of our LP model described in Equation \ref{eq:3.4}. Code is written to generate a file that contains all the candidates' spectral information needed for the test cases. In this step, the molecular property information that generated from the electronic structure calculations are used. For a specific candidate, given the molecular property information and a value $\theta$, the candidate's spectral information is obtained. To further illustrate, a candidate class is written. This class defines how to use the molecular property information to generate the needed spectral information. Given a candidate's molecular property information and a value $\theta$, a instance of this specific candidate is created. For the simplified molecular model, this class only contains IR spectral information. \\

In the second step, additional code is written to generate a target composition of a list of candidates. Then the target composition is used to generate the target spectra. The probe range, which is the range of the wavenumber, is from $2800$  to $3300~\rm cm^{-1}$ for the                    simplified molecular model. It is from $2000$ to $3000~\rm cm^{-1}$ wavenumber for realistic molecules. The target spectral information is generated in the same text file as candidate's spectral information. Depend on the test case setting, code can be used to generate text files that contain selected types of spectral information. \\

In the third step, the LP model is constructed by using the spectral information text file generated in the second step. This part of the code was written by Hung \cite{KuoKaiHung:Thesis:2015}. It reads all the candidates and target spectral information, and builds the LP model as shown in Equation \ref{eq:3.4}. It then outputs our LP input file for LP solver. \\

In the fourth step, we use as LP solver the ``GNU linear progarmming tool kit" (GLPK) which will return the optimum solution for our input file. \\

\section{Test Cases}

In Cases 1 and 2, four candidates are selected. The detailed setting is shown in Table \ref{tab:3.1}. In Case 1, there are four candidates with $\theta$ of $0^{\circ}, 10^{\circ}, 20^{\circ}$, and $30^{\circ}$. In Case 2, the four candidates are of $\theta$ values $0^{\circ}, 5^{\circ}, 10^{\circ}$, and $15^{\circ}$. Instead of having ten degree variance in $\theta$, a five degree difference is applied on $\theta$ in Case 2. This means that in Case 2 the candidates are more similar to each other than the ones in Case 1 as their spectra are more similar. In both cases, 100 data points are selected evenly along the wavenumber from $z$-polarized IR spectra. The target composition of the candidates is the same for both cases. In Case 1, the return composition is the same as the target one, however, the return composition for Case 2 does not match its target. \\

In order to figure out why the return composition in Case 2 is different from the target one, the spectra generated by the return composition is plotted together with the target spectra as shown in Figure \ref{fig:3.2}. Note that the result spectra is identical to the target one. Note that their residual is $0$. In order to see whether this observation can be generalized, Case 3 is set up in Table \ref{tab:3.2}. Case 3 contains more candidates than Cases 1 and 2. Ten candidates are included with $\theta$ values ranging from $0^{\circ}$ to $90^{\circ}$.  \\

Table \ref{tab:3.2} indicates that the return composition of Case 3 is different from the target one. Figure \ref{fig:3.3} shows that the spectrum produced by the return composition is almost identical to the one generated by the target composition in Case 3. The residual is negligible as well. This observation is comparable to Case 2. \\

Among Cases 1, 2 and 3, only the return composition of Case 1 matches its target one. However, in Case 2, the difference in value $\theta$ among the candidates is smaller than Case 1. In Case 3, the number of the candidates is larger than Case 1. Both effects increase the complexity of the test cases. In both Cases 2 and 3, the spectrum constructed by the return composition matches the one built by the target composition. \\

The above observation demonstrates that there are multiple compositions can achieve in constructing the spectrum that are close to the target one. The numerical limitation helps the LP solver to converge to a unique optimum solution. The reason for Case 1 to return a composition that matches the target one, is that the spectral information applied to the LP model is competent. The constraints constructed in the LP model for Case 1 eventually converge to the target composition. \\ 

\begin{table} 
\begin{center}
{\def\arraystretch{1.5}
\begin{tabular}{| l | l | l | }
\hline
Test Case index & 1 & 2  \\
\hline
Number of Candidates & 4 & 4  \\
\hline
Candidates & [0, 10, 20, 30] & [0, 5, 10, 15] \\
\hline
Target Composition & [0.1, 0.5, 0.4, 0] & [0.1, 0.5, 0.4, 0]     \\
\hline
Number of Data Points & 100, $z$ &  100, $z$     \\
\hline
%Return Composition & [0.1, 0.5, 0.4, 0] & [0, 0.796962, 0.103038, 0.1] \\
Return Composition & [0.1, 0.5, 0.4, 0] & [0, 0.80, 0.10, 0.1] \\
\hline
\end{tabular} 
}
\end{center}
\caption{Test cases 1 and 2 for the simplified molecular model.}
\label{tab:3.1}
\end{table}	

\begin{figure}[!ht] 
%\centering
\includegraphics[scale=0.7]{Figures/toy_model_result_plotting_ir_cos_4candi_1.png}
\caption{a. $z$-polarized IR target spectrum plotted with the one constructed by the return composition in Case 2 of simplified molecular model; b. The residual plot between the spectra.}
\label{fig:3.2}
\end{figure}

\begin{table} 
\begin{center}
\begin{tabular}{| l | p{7cm} | }
\hline
Test Case index & 3  \\
\hline
Number of Candidates & 10   \\
\hline
Candidates & [0, 10, 20, 30, 40, 50, 60, 70, 80, 90]  \\
\hline
Target Composition & [0.1, 0, 0.5, 0, 0.4, 0, 0, 0, 0, 0] \\
\hline
Number of Data Points & 100, $z$ \\
\hline
%Return Composition & [0, 0, 0.730541, 0, 0.212061,0, 0, 0.0573978, 0, 0] \\
Return Composition & [0, 0, 0.73, 0, 0.21,0, 0, 0.057, 0, 0] \\
\hline
\end{tabular}
\end{center}
\caption{Test case 3 for the simplified molecular model.}
\label{tab:3.2}
\end{table}	

\begin{figure}[!ht] 
\centering
\includegraphics[scale=0.7]{Figures/toy_model_result_plotting_ir_cos_10candi_1.png} 
\caption{a. $z$-polarized IR target spectrum plotted with the one constructed by the return composition in Case 3 of simplified molecular model; b. The residual plot between the two spectra.}
\label{fig:3.3}
\end{figure}

In order to add necessary information to construct the constraints in our LP model, IR's second polarization is introduced to the simplified molecular model: the $x$ polarization. Figure \ref{fig:3.4} describes how the $x$-polarized spectra presented for $10$ candidates. Case 4 and 5 include both polarizations' spectral information in the LP model. In Table \ref{tab:3.3}, Case 4's setting is based on Case 2, with $x$-polarized IR spectral information added. $100$ data points are selected from this additional spectrum, then converted to additional decision variables and constraints in the LP model. Case 5 is based on Case 3, with $x$-polarized IR spectral information added. In both Case 4 and 5, the return composition matches the target one. This further demonstrates that as long as we have sufficing information for the LP model, the LP solver returns a composition matches the target one. \\ 

\begin{figure}[!ht]
\centering
\includegraphics[scale=0.7]{Figures/Toy_Model_IR_Sine_Projection.png} 
\caption{$x$-polaried IR spectra of candidates of simplified molecular model with $\theta$ value expanded from $0^{\circ}$ to $90^{\circ}$.}  \label{fig:3.4}
\end{figure}

\begin{table} \small 
\begin{center}
{\def\arraystretch{1.5}
\begin{tabular}{| l | p{3cm} | p{6cm} |}
\hline
Test Case index & 4 & 5\\
\hline
Number of Candidates & 4 & 10 \\
\hline
Candidates & [0, 5, 10, 15] & [0, 10, 20, 30, 40, 50, 60, 70, 80, 90] \\
\hline
Target Composition & [0.1, 0.5, 0.4, 0] & [0.1, 0, 0.5, 0, 0.4, 0, 0, 0, 0, 0]\\
\hline
Number of Data Points & 100, $z$ \newline 100, $x$ & 100, $z$ \newline 100, $x$\\
\hline
Return Composition & [0.1, 0.5, 0.4, 0] & [0.1, 0, 0.5, 0, 0.4, 0, 0, 0, 0, 0] \\
\hline
\end{tabular}
} 
\caption{Test cases 4 and 5 for the simplified molecular model.}\label{tab:3.3}
\end{center}
\end{table}		

\section{Constraint Study Based on Test Case 4}

From Cases 1 to 5 for our simplified molecular model, we know that having an instance with sufficient information as input to our LP model is the key to obtain the target composition. Having sufficient information means having enough constraints to help to converge to the desired target composition. The information stems from the data points selected along the spectra. This leads us to do a more detailed study on the constraints in order to see how many data points are enough to get the target composition.\\ 

Based on Case 4, test cases about creating different LP instances using different spectral information are designed in Table \ref{tab:3.4}. The number of data points indicates how many data points are selected. Points Selection shows how data points are selected. For example, [2800, 3300, 50] means along wavenumber from 2500 to 3300, every 50 wavenumber a data point is selected along a spectrum. $z$ and $x$ indicate the corresponding polarization of IR spectrum. \\

\begin{table} \small
\begin{center}
{\def\arraystretch{1.5}
\begin{tabular}{| p{3cm} | p{3cm} | p{4cm} | l |} \hline
	Test Case \# & \# Data Points & Points Selection & Return Composition \\ \hline
	6 & 10 & [2800, 3300, 50], $z$ & [0, 0.8, 0.10, 0.1] \\ \hline
	7 & 20 & [2800, 3300, 25], $z$ & [0, 0.8, 0.10, 0.1 \\ \hline
	8 & 25 & [2800, 3300, 20], $z$ & [0, 0.8, 0.10, 0.1] \\ \hline
	9 & 32 & [2800, 3300, 15], $z$ & [0, 0.8, 0.10, 0.1] \\ \hline
	10 & 50 & [2800, 3300, 10], $z$ & [0, 0.8, 0.10, 0.1] \\ \hline
	11 & 100 & [2800, 3300, 5], $z$ & [0, 0.8, 0.10, 0.1] \\ \hline
	12 & $100 + 1$ & [2800, 3300, 5], $z$ \newline [2800, 3300, 500], $x$ & [0, 0.8, 0.10, 0.1] \\ \hline
	13 & $100 + 5$ & [2800, 3300, 20], $z$ \newline [2800, 3300, 100], $x$ & [0, 0.8, 0.10, 0.1] \\ \hline
	14 & $100 + 10$ & [2800, 3300, 20], $z$ \newline  [2800, 3300, 50], $x$ & [0, 0.8, 0.10, 0.1] \\ \hline
	15 & $100 + 50$ & [2800, 3300, 20], $z$ \newline  [2800, 3300, 10], $x$ & [0.1, 0.5, 0.4, 0] \\ \hline
	16 & $100 + 100$ & [2800, 3300, 20], $z$ \newline  [2800, 3300, 5], $x$ & [0.1, 0.5, 0.4, 0] \\ 
	\hline
\end{tabular} 
}
\end{center}
\caption{Constraint study based on Case 4 for the simplified molecular model. For more detailed result data, refer to Table \ref{tab:7.3}.} \label{tab:3.4}
\end{table}

As Table \ref{tab:3.4} indicates, the return compositions in Cases 6 to 14 do not return the target compostion. To the contrary, in Cases 15 to 16, the return composition matches the target one. Figure \ref{fig:3.5} displays the spectra conducted by $[0, 0.80, 0.1, 0.1]$  and $[0.1, 0.5, 0.4, 0]$, both $x$- and $z$-polarized IR spectra generated by these two compositions are identical.
%$[0, 0.796962, 0.103038, 0.1]$

\begin{figure}[!ht] 
\centering
\includegraphics[scale=0.7]{Figures/toy_model_result_plotting_ir_sin_4candi_constraint_study_experiment4.png} 
\caption{IR spectra plotted by the return compositions from the constraint study based on Case 4 of simplified molecular model. a. $z$-polarized IR spectra; b. $x$-polarized IR spectra.}\label{fig:3.5}
\end{figure}


\section{Constraint Study Based on Test Case 5}

Based on Case 5, similar constraint study is conducted as displayed in Table \ref{tab:3.5}, and the same observation is obtained as the test cases in Table \ref{tab:3.4}. When the result composition $[0, 0, 0.73, 0, 0.21,0, 0, 0.057, 0, 0]$ and target one are used to plot the spectra, the produced spectra are identical, as shown in Figure \ref{fig:3.6}. Although these two constraint studies do not give a clear answer about how many data points are enough to get the target composition, it confirms that as long as the spectral information is sufficient, the LP solver will return the target composition.
%$[0, 0, 0.730541, 0, 0.212061,0, 0, 0.0573978, 0, 0]$
\begin{table} \small
\begin{center} 
{\def\arraystretch{1.5}
\begin{tabular}{| p{1cm} | p{2cm} | p{4cm}  | l |}
\hline
Test Case \# & \# of Data Points & Point Selection & Return Composition \\ \hline
17 & 10 & [2800, 3300, 50], $z$ & [0.16, 0, 0, 0.83, 0, 0, 0, 0, 0, 0.017] \\ \hline
18 & 25 & [2800, 3300, 20], $z$ & [0, 0, 0.73, 0, 0.21, 0, 0, 0.057, 0, 0, 0] \\ \hline
19 & 50 & [2800, 3300, 10], $z$ & [0, 0, 0.73, 0, 0.21, 0, 0, 0.057, 0, 0, 0] \\ \hline
20 & 100 & [2800, 3300, 5], $z$ & [0, 0, 0.73, 0, 0.21, 0, 0, 0.057, 0, 0, 0] \\ \hline
21 & 500 & [2800, 3300, 1], $z$ & [0, 0, 0.73, 0, 0.21, 0, 0, 0.057, 0, 0, 0] \\ \hline	
22 & $100 + 1$ & [2800, 3300, 5], $z$ \newline [2800, 3300, 500], $x$  & [0, 0, 0.73, 0, 0.21, 0, 0, 0.057, 0, 0, 0] \\ \hline
23 & $100 + 10$ & [2800, 3300, 5], $z$ \newline [2800, 3300, 50], $x$  & [0.36, 0, 0.31, 0.33, 0, 0, 0, 0, 0] \\ \hline
24 & $100 + 20$ & [2800, 3300, 5], $z$ \newline [2800, 3300, 25], $x$  & [0.17, 0, 0, 0.79, 0, 0, 0.035, 0, 0, 0] \\ \hline
25 & $100 + 25$ & [2800, 3300, 20], $z$ \newline [2800, 3300, 20], $x$  & [0.17, 0, 0, 0.79, 0, 0, 0.035, 0, 0, 0] \\ \hline
26 & $100 + 50$ & [2800, 3300, 5], $z$ \newline [2800, 3300, 10], $x$  & [0, 0, 0.75, 0, 0.15, 0, 0.1, 0, 0, 0] \\ \hline
27 & $100 + 84$ & [2800, 3300, 5], $z$ \newline [2800, 3300, 6], $x$  & [0.17, 0, 0, 0.79, 0, 0, 0.035, 0, 0, 0] \\ \hline
28 & $100 + 100$ & [2800, 3300, 5], $z$ \newline [2800, 3300, 5], $x$  & [0.1, 0, 0.5, 0, 0.4, 0, 0, 0, 0, 0] \\ 
\hline
\end{tabular} \\
}
\caption{Constraint study based on Case 5 of simplified molecular model. For more detailed result data, refer to Table \ref{tab:7.4}.}\label{tab:3.5}
\end{center}
\end{table}

\begin{figure}[!ht] 
\centering
\includegraphics[scale=0.7]{Figures/toy_model_result_plotting_ir_sin_10candi_constraint_study_experiment5.png} 
\caption{IR spectra plotted by the return compositions from the constraint study based on Case 5 of simplified molecular model. a. $z$-polarized IR spectra; b. $x$-polarized IR spectra.}\label{fig:3.6}
\end{figure}

\section{Discussion and Conclusion}

Recall that our LP model, for sufficient data sets are expected to return the target composition. We can conclude that, if the target composition is not returned correctly, then the data we collect is not sufficient to describe the test cases to the LP model. \\

However, when the target composition is not returned correctly, the return composition does build spectra that are identical to the target ones. This means that there is more than one composition that can build the spectra that are identical to the target ones. \\

With the help of the simplified molecular model, we know the reason why the LP model cannot return the target composition in some test cases. In the next step, we want to figure out with all the spectral information available for a realistic molecular model, can the  instances of our LP model return the target composition?

%The above conclusion leaves us a new question: how do we know there is sufficient spectral information in order to obtain the target composition of candidates at interfaces? To answer this question, further test cases are conducted by applying the spectral information of realistic molecules to the LP model. The goal is to investigate with all the spectral information we can obtain for realistic molecules, can the LP model return the target composition of candidates at interfaces. If this goal can be achieved, can this approach be applied systematically to different circumstances?

%Above analysis simulates the following question: how can we know there is enough information to achieve the target composition? In the next step, we will experiment with real molecules. The goal is to investigate with all the spectral information that we can obtain for real molecules, can our LP model return the target composition for the target spectrum? If yes, can we apply the LP model systematically? Furthermore, to maximally explore the capacity of our LP model, and study its limitation. Finally, come up with some general instructions for applying our LP model. These are the main focus for the following chapters.\\

		

		 



