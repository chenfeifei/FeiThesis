\startfirstchapter{Introduction}
\label{chapter:introduction}
\section{Structure of molecules adsorbed to surfaces}

Any pictures?

\section{Experimental probes: IR, Raman, SFG}
(Give a brief intro. for people from CS)


\section{Current approaches to structure elucidation}
(Lots reference can go here, previous work)


\section{Linear programming}
(Explain it to Chemistry Department)
LP formula: objective function, constraint

complex algorithm

Three kinds of Soluction

Why we select LP as the tool to tackle this problem? The advantage of LP for this problem. Is there any previous work related?

Introduce the simplex method, the toolkit.
Linear programming can deal with thousands of variables.\\

Linear programming is a convex, deterministic process, and it is guaranteed to converge to a single global optimal if there is a solution space. Linear Programming problems are intrinsically easier to solve than general nonlinear problems.\\

For a linear programming problem, it is by nature that there are three kinds of situations may happen.\\

First, there is no feasible solution.\\

Second, there are feasible solutions, and the solutions space is bounded in a polyhedron, and only one globally optimal solution(either a single point or multiple equivalent points along a line) exists. Furthermore, in this unique and convex feasible region, the optimal solution is at one of the vertex of this convex polygon.\\

Third, the solution space is unbounded, there are multiple solutions, however, no optimal solution for the existing problem.\\

Because of this property, it is guaranteed that if linear programming solver returns a solution, then this solution is the global optimal one for the problem.\\

Simplex method:
Introduce Linear Programming solver:


	
	
	

	
	
	
	
	
	
Compare linear programming with quadratic programming, why linear programming is a better 	 	approach to the problem? (Having problems finding related work or how to prove it myself)
	
	Is there only computational gain?
	Also consider the model itself and solution space	
(The problem is defined as "Candidate ratio problem" in Kai's thesis, same here???)
	to determine the level of similarity between spectra is not an easy task 
	
	How should I introduce Linear Programming here?
	The advantages of linear programming are: 
\section{Aims and scope}
What's LP's constraint?

