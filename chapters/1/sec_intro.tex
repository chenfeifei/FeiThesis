\section{Introduction}
\subsection{Background and Motivation}


For current molecules that we are studying, what surfaces do you assume here?
We are not referring to any specific surfaces here.


	Explain the problem: 
	What is the study of molecule orientation at surface?
	What are the modern techniques? 
	What are spectroscopy techniques? 
	IR
	Ramam
	SFG
	
	
	those three spectroscopic measurements are all non-destructive, relatively simple and can be performed in situ for a wide range of the "surrounding/film/substrate" systems.\cite{01}\cite{02} \\
	
	IR and Raman are linear optical processes, no surface specificity, sensitive to molecular orientation in even orders. SFG is nonlinear optical process, surface specificity(non-centrosymetric arrangement), sensitive to molecular orientation in odd orders
		
	IR: $\big<P_2\big> = \frac{1}{2} \big<3\cos^2\theta -1 \big> $ \\
	
	Raman: $\big<P_2\big> = \frac{1}{2} \big<3\cos^2\theta -1 \big> $, $\big<P_4\big> = \frac{1}{8} \big< 35\cos^4\theta -30\cos^2\theta + 3 \big> $ \\
	
	SFG: $\big<P_1\big> = \big< \cos\theta \big> $,
	$\big<P_3\big> = \frac{1}{2} \big<5\cos^3\theta -3\cos\theta \big> $ \\


IR stands for Infrared Resonant, its absorption is absorption-transmission-reflexion mode(resonant), the physical principle is variation of the dipolar moment along the normal coordinates: $\frac{\partial\mu}{\partial Q} $, the first rank tensor $ \mu $. There are 3 elements, x, y, z.\\

Raman is scattering from sample, non resonant, is the variation of the molecular polarizability along the normal coordinates: $\frac{\partial\alpha}{\partial Q} $, the second rank tensor $\alpha$. There are 9 elements, xx, yy, zz, xy, xz, yx, zx, yz and zx.\\

SFG stands for sum-frequency generation spectroscopy, it is reflexion but still resonant, it is variation of both the dipolar moment and the polarizability, $\frac{\partial\mu}{\partial Q} *  \frac{\partial\alpha}{\partial Q} $, the third rank tensor $ \chi^{(2)}$. There are 27 elements, Rubik's cube. \\

Comparing to linear optical spectroscopy, the biggest advantage of SFG is that it is surface specific $ $, P and E are vectors here. In bulk, there is inversion symmetry, E and P change sign under inversion, therefore,$
\chi^{(2)} = 0$. However, at surfaces, there is no symmetry inversion, so that $\chi^{(2)} \neq 0 $
	
	
	
	What are the advantages of using spectroscopy techniques? \\
	What are the advantages and disadvantages of each spectroscopy technique? \\
	Why do we want to compare those spectroscopy techniques? \\
	Introduce toy model?


