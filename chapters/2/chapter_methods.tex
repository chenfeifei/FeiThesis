\startchapter{Methods} \label{ch:2}
\section{Description}
Before introducing and analyzing the LP model and applying it to the realistic molecules' vibrational spectra, there are a few factors to address. First of all, creating each amino acid's IR, Raman and SFG spectra is an essential step. This part research has been done thoroughly by Hung \cite{KuoKaiHung:Thesis:2015}. In this chapter, I introduce the content that is related to our study. \\

\section{Structure of Realistic Molecules}
Figure \ref{fig:aminacids} illustrates the molecule structure of the six amino acids in the molecular frame. These amino acids are used in the test cases related to realistic molecules. The $a$, $b$ and $c$ are the molecular frame coordinates. When a molecule lays on a surface, we need to transfer the molecular frame to the lab frame where the surface exists. \\

\begin{figure}
\begin{subfigure}{.5\textwidth}
  \centering
  \includegraphics[height=5cm]{Figures/Ala.png}
  \caption{Ala}
  \label{fig:sfig1}
\end{subfigure}%
\begin{subfigure}{.5\textwidth}
  \centering
  \includegraphics[height=5cm]{Figures/Met.png}
  \caption{Met}
  \label{fig:sfig2}
\end{subfigure}
\begin{subfigure}{.5\textwidth}
  \centering
  \includegraphics[height=5cm]{Figures/Thr.png}
  \caption{Thr}
  \label{fig:sfig3}
\end{subfigure}
\begin{subfigure}{.5\textwidth}
  \centering
  \includegraphics[height=5cm]{Figures/Leu.png}
  \caption{Leu}
  \label{fig:sfig4}
\end{subfigure}
\begin{subfigure}{.5\textwidth}
  \centering
  \includegraphics[height=5cm]{Figures/Ile.png}
  \caption{Ile}
  \label{fig:sfig5}
\end{subfigure}
\begin{subfigure}{.5\textwidth}
  \centering
  \includegraphics[height=5cm]{Figures/Val.png}
  \caption{Val}
  \label{fig:sfig6}
\end{subfigure}
\caption{Molecule structure of Ala, Met, Thr, Leu, Ile and Val in molecular frame. Blue axis is designated as $c$ axis, red axis is designated as $a$ axis, green is designated as $b$ axis.}
\label{fig:aminacids}
\end{figure}

\section{Generating Model Spectra \cite{ss}}
To generate these amino acids' vibrational spectra, a molecule's vibration modes need to be modelled in the molecular frame, and then transferred to the laboratory frame where surfaces exist. Chapter 2 in Hung's thesis \cite{KuoKaiHung:Thesis:2015} describes how to perform electronic structure calculations using a software package called The General Atomic and Molecular Electronic Structure System (GAMESS) \cite{GAMESS} to obtain the derivatives of every dipole moment and polarizability. Then he introduced how to use Direction Cosine Matrix (DCM) to transfer these two derivatives from the molecular frame to the laboratory one. After that, Euler angles could be extracted from DCM. Euler angles are used to describe a molecule's orientation at surfaces. They are labelled by $\theta$, $\varphi$ and $\psi$ as shown in Figure \ref{fig:2.1}. They are referred to as $tilt$, $azimuthal$ and $twist$ angles, respectively. Let $x$, $y$ and $z$ be lab frame Cartesian coordinates, and let $a$, $b$ and $c$ be the molecular frame coordinates. $Tilt$ angle $\theta$ is the angle between $z$ and $c$. $Azimuthal$ angle $\varphi$ is the rotation about $z$. $Twist$ angle $\psi$ is a twist about $c$ \cite{hore0033-rotations}. After three steps of successive rotations of Euler angles, molecule properties can be transferred from the molecular frame to the lab one. \\

In order to achieve the above steps, Hung first did a Hessian calculation using GAMESS. Secondly, seven snapshots of a molecule vibrating in different modes were taken. Thirdly, he did a force field calculation to obtain the derivatives of dipole moment and polarizability for each of the seven snapshot moment. Then the derivatives of dipole moment and polarizability are obtained by the interpolation of these seven snapshot moment. Because the two obtained derivatives are in the molecular frame, Hung used DCM to convert these two derivatives into the lab frame. Then he abstracted Euler angles from DCM. After these electronic structure calculations, the derivatives information, which is the molecular property information, is obtained. \\

\begin{figure}[!ht] 
\centering
\includegraphics[scale=0.5]{Figures/Euler_angles_represented_as_the_spherical_polar_angles.png} 
\caption{The Euler angles represented as the spherical polar angles $\theta$, $\varphi$ and $\psi$, and the illustration of the three successive rotations that transform the lab $x$, $y$, $z$ coordinate system into the molecular $a$, $b$, $c$ frame intrinsically and extrinsically. Reproduced from Ref. \citenum{hore0033-rotations}. }
\label{fig:2.1}
\end{figure}

In our study, those molecular property information is used to generate the amino acids' spectral information directly. Each molecule's property information contains the derivatives of dipole moment and polarizabilities of each vibrational mode. Depending on the number $N$ of atoms in a molecule, there are $3N-6$ vibrational modes. Furthermore, Equations \ref{eq:2.2} to \ref{eq:2.4} are used to generate the amino acids' IR, Raman and SFG spectra. \\

All the test cases in our study are limited to only consider the $tilt$ angle distribution of Euler angles, and assume isotropy on $twist$ and $azimuthal$ angular distributions. A uniform distribution is applied to $twist$ and $azimuthal$ angles. For angle $\varphi$, it requires the surfaces to be not striped, so that the molecule has no preference on the $xy$ plane on the lab frame. There can be no anisotropy in the plane of the surface. Because of this, we can limit the candidate number by integrating angle $\varphi$. For angle $\psi$, a uniform distribution implies that a molecule has cylindrical symmetry in its preference of surface. The molecule can be tilted, but has no `$twist$' preference. With the integration of these two Euler angles, the number of candidates for one molecule will be greatly reduced. Therefore, a candidate in our study is a specific molecule with specific $\theta$ value. However, the number of the candidates is still large when considering angle $\theta$ only. For example, from $0^{\circ}$ till $180^{\circ}$, candidates are obtained in $10^{\circ}$ steps, there are $18$ candidates for just one molecule. For a mixture of six molecules, the number of possible combinations of all these molecules' candidates is $18^{6} = 34012224$. \\

When molecules lay on a surface, the orientation of each single molecule varies. To simulate the vibrational spectra, a reasonable orientation distribution for the molecules needed to be studied. The orientation distribution requires either to do a molecular dynamic simulation in order to study the distribution of molecule orientations at surface, or come up with an analytic orientation distribution function. In our study, the LP approach is appropriate for the second method. Moreover, the $\delta$-distribution function shown in Equation \ref{eq:2.1} is used to represent the molecule orientation distribution that models the spectrum signals. This means that all the molecules are tilted at the same angle at surface. This assumption is applied across the whole study. \\

\begin{equation} \label{eq:2.1}
f_{(\theta)} = \delta(\theta - \theta_{o})
\end{equation} 

The absorption of an IR spectrum is proportional to the square of the lab-frame dipole moment derivative. For example, the $x$-polarized absorption spectrum is given by Equation \ref{eq:2.2}: \\

\begin{equation} \label{eq:2.2}
A_x(\omega_{\rm IR}) = \sum_{q} \frac{1}{2 m_{q} \omega_{\rm q}} \Bigg \langle \Bigg\lbrack \frac{\partial u_x}{\partial Q} \Bigg\rbrack_{q} ^2 \Bigg\rangle \frac{\Gamma_q^2}{(\omega_{\rm IR}-\omega_{\rm q})^2 + \Gamma_q^2}
\end{equation} 
where $A_x$ represents $x$-polarized IR obsorbance. $\omega_{\rm IR}$ is the frequency of the probe radiation, $\mu$ is the dipole moment, $m_q$ is the reduced mass, $w_q$ is resonance frequency. $\Gamma_q$ is the homogeneous line width, is set to $6$ in all the test cases. $Q_q$ is the normal mode coordinate of the $q$th vibrational mode. All values of $\omega_{\rm IR}$, $\mu$, $m_q$, $Q$ are obtained from the electronic structure calculations. Furthermore, because $\varphi$ and $\psi$ angles are integrated, the $x$-polarized spectrum is identical with the $y$-polarized one. Therefore, there are only two unique polarized IR spectra. For simplicity, IR spectra are referred to as $x$ and $z$ in future test cases. $A_y$ and $A_z$ are computed accordingly.\\

The intensity of Raman scattering is proportional to the square of lab frame transition polarizability. For example, Raman spectroscopy with an $x$-polarized excitation source collects the $x$-polarized component of the scattered radiation, which can be approximated using Equation \ref{eq:2.3}. \\

\begin{equation} \label{eq:2.3}
I_{xx}(\Delta \omega) = \sum_{q} \frac{1}{2 m_{q} \omega_{\rm q}} \Bigg \langle \Bigg\lbrack \frac{\partial \alpha_{xx}^{(1)}}{\partial Q} \Bigg\rbrack_{q} ^2 \Bigg\rangle \frac{\Gamma_q^2}{(\Delta \omega-\omega_{\rm q})^2 + \Gamma_q^2}
\end{equation} 
where $I_{xx}$ represents $xx$-polarized Raman intensity. $\Delta w$ is the Stokes Raman shift. $\alpha_{xx}^{(1)}$ is one component of the nine-element polarizability tensor. $m_q$, $w_q$, $\Gamma_q$, and $Q_q$ are the same as defined above for IR spectra. All the values of $\omega_{\rm IR}$, $\mu$, $m_q$, $Q$ are obtained from the electronic structure calculations. Similar to IR spectroscopy, because of the integration of $\varphi$ and $\psi$ angles, only four unique spectra are obtained from the following polarization: $xx$, $xy$, $xz$ and $zz$. For simplicity, Raman spectra are referred to as $xx$, $xy$, $xz$ and $zz$ in future test cases. \\

%The intensity of SFG spectroscopy is proportional to the squared magnitude of the second-order susceptibility, $\left|\chi^{(2)}\right|^{2}$. $\chi^{(2)}$ is derived from the second-order polarizability, $\alpha^{2}$. 
SFG spectral signal is the imaginary part of the second-order susceptibility, $\left|\chi^{(2)}\right|$. $\chi^{(2)}$ is derived from the second-order polarizability $\alpha^{(2)}$ as shown in Equation \ref{eq:kai}. The imaginary part of $\left|\chi^{(2)}\right|$, which is the SFG spectral signal, is displayed as Equation \ref{eq:2.4}.

\begin{equation} \label{eq:kai}
\chi^{(2)}_{xxz} (\omega_{\rm IR}) = \sum_{q} \frac{1}{2 m_{q} \omega_{\rm q}} \Bigg \langle \Bigg\lbrack \frac{\partial \alpha_{xx}^{(1)}}{\partial Q} \Bigg\rbrack_{q} \Bigg\lbrack \frac{\partial u_{z}}{\partial Q} \Bigg\rbrack_{q} \Bigg\rangle \frac{1}{\omega_{\rm q}-\omega_{\rm IR}+i\Gamma_q}
\end{equation} 

\begin{equation} \label{eq:2.4}
\rm Im \Bigg\lbrack \chi^{(2)}_{xxz} (\omega_{\rm IR}) \Bigg\rbrack= \sum_{q} \frac{1}{2 m_{q} \omega_{\rm q}} \Bigg \langle \Bigg\lbrack \frac{\partial \alpha_{xx}^{(1)}}{\partial Q} \Bigg\rbrack_{q} \Bigg\lbrack \frac{\partial u_{z}}{\partial Q} \Bigg\rbrack_{q} \Bigg\rangle \frac{\Gamma_q}{(\omega_{\rm q}-\omega_{\rm IR})^{2}+\Gamma_q^{2}}
\end{equation} 
where $\chi^{(2)}_{xxz}$ is the second-order susceptibility. It is probed by an $x$-polarized visible incoming beam at frequency $\omega_{\rm vis}$ and a $z$-polarized infrared beam incoming with frequency $\omega_{\rm IR}$. Both incoming beams are incident to the sample. Then the $x$-component at frequency $\omega_{\rm SFG}=\omega_{\rm vis}+\omega_{\rm IR}$ is selected for detection. As $i=\sqrt{-1}$ is in the denominator, $\chi^{(2)}$ is a complex value \cite{KuoKaiHung:Thesis:2015}. The SFG response considered in this thesis is the imaginary component of the $\chi^{(2)}$. Same as IR and Raman spectroscopy, all the values of $\omega_{\rm IR}$, $\mu$, $m_q$, $Q$ are obtained from the electronic structure calculations. Because of the integration of $\varphi$ and $\psi$ angles, only three unique non-zero spectra are obtained from the following polarizations: $xxz$, $xzx$ and $zzz$. For simplicity, SFG spectra are referred as $xxz$, $xzx$ and $zzz$ in future test cases. \\

With these equations and the electronic structure calculations, IR, Raman and SFG spectra can be generated for a candidate of a molecule. Taking Met as an example, Figure \ref{fig:2.2} displays $x$-polarized IR spectra of the following candidates: Met with $\theta$ equals $0^{\circ}$, $20^{\circ}$, $40^{\circ}$ and $60^{\circ}$. Their spectra are prefixed with $candidate\_$ in the labels. $ir\_x\_$ indicates the spectroscopy technique, ``number" indicates the $\theta$ angle's value. The spectra labelled as $target\_ir\_x$ is generated by combining $10\%$ of $candidate\_ir\_x\_0$, $50\%$ $candidate\_ir\_x\_20$ and $40\%$ $candidate\_ir\_x\_40$. \\

Similarly, Figures \ref{fig:2.3}, \ref{fig:2.4} and \ref{fig:2.5} depict the spectra of the same candidates and targets for $z$-polarized IR, $xx$-polarized Raman and $xxz$-polarized SFG spectrum, respectively. In Figure \ref{fig:2.2}, the biggest differences among the candidates exist at each vibrational mode. The valid range for the wavenumber is $1000$ to $2000$. \\

\begin{figure}[!ht]
\centering
\includegraphics[scale=0.5]{Figures/Met_candidates_plotting_ir_x.png}
\caption{IR $x$-polarized spectra of methionine's four candidates and target. The candidates are with $\theta$ of $0^{\circ}$, $20^{\circ}$, $40^{\circ}$ and $60^{\circ}$. The target is produced by combining $10\%$ of $candidate\_ir\_x\_0$, $50\%$ $candidate\_ir\_x\_20$ and $40\%$ $candidate\_ir\_x\_40$. } \label{fig:2.2}
\end{figure}

\begin{figure}[!ht]
\centering
\includegraphics[scale=0.5]{Figures/Met_candidates_plotting_ir_z.png}
\caption{IR $z$-polarized spectra of methionine's four candidates and target. The candidates are with $\theta$ of $0^{\circ}$, $20^{\circ}$, $40^{\circ}$ and $60^{\circ}$. The target is produced by combining $10\%$ of $candidate\_ir\_x\_0$, $50\%$ $candidate\_ir\_x\_20$ and $40\%$ $candidate\_ir\_x\_40$.} \label{fig:2.3}
\end{figure}

\begin{figure}[!ht]
\centering
\includegraphics[scale=0.5]{Figures/Met_candidates_plotting_raman_xx.png}
\caption{Raman $xx$-polarized spectra of methionine's four candidates and target. The candidates are with $\theta$ of $0^{\circ}$, $20^{\circ}$, $40^{\circ}$ and $60^{\circ}$. The target is produced by combining $10\%$ of $candidate\_ir\_x\_0$, $50\%$ of $candidate\_ir\_x\_20$ and $40\%$ of $candidate\_ir\_x\_40$.} \label{fig:2.4}
\end{figure}

\begin{figure}[!ht]
\centering
\includegraphics[scale=0.5]{Figures/Met_candidates_plotting_sfg_xxz.png}
\caption{SFG $xxz$-polarized spectra of methionine's four candidates and target. The candidates are with $\theta$ of $0^{\circ}$, $20^{\circ}$, $40^{\circ}$ and $60^{\circ}$. The target is produced by combining $10\%$ of $candidate\_ir\_x\_0$, $50\%$ of $candidate\_ir\_x\_20$ and $40\%$ of $candidate\_ir\_x\_40$.} \label{fig:2.5}
\end{figure}

\section{Conclusion}
Chapter \ref{ch:2} briefly explains what the current approaches are to extract molecular orientation distribution at surfaces, the molecular structures of six amino acids, and how to produce IR, Raman and SFG spectra theoretically. In Chapter \ref{ch:3}, our LP model is described and its properties are studied. It is conducted by using a simplified molecular model to gain an insight of our approach. The motivation of creating a simplified molecular model is to create a molecule as simple as possible that will allow us to study the properties of the LP model for this basic case. Information gained in Chapter \ref{ch:3} allows us to then study the approach using molecules in Chapters \ref{ch:4}, \ref{ch:5} and \ref{ch:6}.