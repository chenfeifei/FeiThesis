\startchapter{Conclusion and Future Work}  \label{ch:7}

\section{Conclusion}
In addition to traditionally exhaustive way in studying the possible composition of candidates of model spectra, the use of LP has been explored by Hung \cite{KuoKaiHung:Thesis:2015}. It has been shown that LP can solve this problem in pseudo polynomial time $O(n)$, which is much better compared to $O(n!)$. However, the reason why the LP model does not always return the target composition of mock spectra was unknown. The first goal of our study is to figure out this reason. \\

To achieve the first goal, a simplified molecular model is designed to analyze the nature of the LP model. It has shown that as long as there is the right data set, the target composition is obtained. If the target composition is not returned correctly, then there is not sufficient spectral information to describe the test cases to the LP model. \\

Furthermore, when we use all the three types of spectral information of a realistic molecule (Met) to build the LP instances, it is not guaranteed to return the target composition. The spectral information we collect is not sufficient in describing the test cases to the LP model most of the time. \\

%With a detailed analysis of applying toy model IR spectral information to our LP model, we have learnt that the reason why our LP model does not return the target composition is that: the spectral data we extract does not always contain sufficient information in order for the LP model to converge to the target composition. As long as the data we collect is sufficient, the LP model guarantees to return a composition we expect. \\

%Based on this observation, we explore various cases. First of all, the case with candidates coming from type of molecule at interface is studied. In this case, the LP model cannot return a composition that match the target one most of the time. It is proved that is because the spectral data applied to the LP model does not contain sufficient information to obtain the target composition. \\

Following the scenario of having one type of realistic molecule at surfaces, the test cases of having multiple types of realistic molecules at surface are also explored. When each molecule's candidates expanded in $[0^{\circ}$, $90^{\circ})$ on $\theta$, Raman or SFG spectral information alone carries sufficient data set in order to obtain the target composition. When the candidates are expanded in $[0^{\circ}$, $180^{\circ}]$ on $\theta$, excluding $90^{\circ}$, SFG spectral information needs to combine with IR or Raman in order to obtain sufficient data set. \\

When comparing the experimental spectra of IR, Raman or SFG techniques with the candidate spectra generated mathematically, there is an amplitude scaling factor. Therefore, in the following test cases, we also consider this scaling factor in the LP instances built by using different spectral information. When consider a mixture of realistic molecules with candidates expanded in $[0^{\circ}$, $90^{\circ})$ on $\theta$, Raman spectral information alone contains sufficient data set in order to obtain the target composition. As the target composition can be re-constructed from the return SV and composition. The SF equals 1 minus SV. When each realistic molecule's candidates are expanded in $[0^{\circ}$, $180^{\circ}]$ on $\theta$, excluding $90^{\circ}$, Raman and SFG spectral information together contains sufficient data set. \\

\section{Contributions}
\begin{itemize}
  \item By studying the LP instances built by using spectral information of simplified molecular model, the inability to extract a sufficient data set from the available spectral information to build the LP instances is the reason that the LP solver does not return the target composition.
  \item In the case of studying one type of realistic molecular model at surfaces, even combining all three spectral information of IR, Raman and SFG, the extracted data set is still not sufficient for the LP instances to obtain the target composition in most cases. 
  \item In the case of different types of realistic molecular models at surfaces, Raman or SFG spectral information alone carries sufficient data set in order to obtain the target composition when candidates of each molecular model expanded in $[0^{\circ}$, $90^{\circ})$ on $\theta$. When candidates of each molecular model expanded in $[0^{\circ}$, $180^{\circ}]$ on $\theta$, excluding $90^{\circ}$, SFG spectral information needs to be combined with IR or Raman in order to obtain a sufficient data set.
  \item When the scaling factor is introduced to each spectral technique, the case of different types of realistic molecular models at surfaces is considered. When each molecular model's candidates expanded in $[0^{\circ}$, $90^{\circ})$ on $\theta$, Raman spectral information alone still carries sufficient data set in order to obtain the target composition. When each molecular model's candidates expanded in $[0^{\circ}$, $180^{\circ}]$ on $\theta$, excluding $90^{\circ}$, Raman and SFG spectral information together carries sufficient data set in order to obtain the target composition.
\end{itemize}

\section{Future Work}
Our LP model has proven its efficiency in studying molecular orientation distribution at surfaces when different molecules are considered. However, when considering one type of molecule at the surface, the spectral information does not carry sufficient data set in order to obtain the target composition. One of the most important direction is to develop new lab methods to collect more sufficient data set for our LP model.

Moreover, checking how scalable our LP model for the cases having candidates coming from different molecules is another improve direction. The number of molecules can be larger. Meanwhile, different types of molecules should also be tested as well.

At last not the least, comparing LP technique with other computation model to further study the solution space is an interesting direction. 










