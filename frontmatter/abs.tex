\newpage
\TOCadd{Abstract}

\noindent \textbf{Supervisory Committee}
\tpbreak
\panel

\begin{center}
\textbf{ABSTRACT}
\end{center}

%Applying linear programming to vibrational spectra to extract the molecular structure at interfaces is a new approach. Research has been done to explore the possibility of using linear programming. However, this new approach has not been studied systematically. We show the use of linear programming using spectral information data first for a toy model, then for situations with real molecules. Appropriateness for the use of the data from IR, Raman, and SFG spectroscopy techniques for our use case is discussed.

Applying linear programming to spectroscopy techniques, such as IR, Raman and SFG, is a new approach to extract the molecular orientation information at surfaces. Research has shown the computational gain when using the linear programming approach. However, linear programming approach does not always return the known molecular orientation distribution information when mock spectral information is applied to the linear programming model. The goal of my study is to figure out the reason that causes the failure. To achieve this goal, a simplified molecular model is designated to study the nature of the linear programming model. With the information gained, I further apply the linear programming approach to various cases in order to verify whether it can be systematically applied in different circumstances.  