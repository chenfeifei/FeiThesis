\newpage
\TOCadd{Abstract}

\noindent \textbf{Supervisory Committee}
\tpbreak
\panel

\begin{center}
\textbf{ABSTRACT}
\end{center}

%Applying linear programming to vibrational spectra to extract the molecular structure at interfaces is a new approach. Research has been done to explore the possibility of using linear programming. However, this new approach has not been studied systematically. We show the use of linear programming using spectral information data first for a toy model, then for situations with real molecules. Appropriateness for the use of the data from IR, Raman, and SFG spectroscopy techniques for our use case is discussed.

Applying linear programming to spectroscopy techniques, such as IR, Raman and SFG, is a new approach to extract the molecular orientation information at surfaces. In Hung's previous research, he has shown how applying linear programming results in the computational gain from $O(n!)$ to $O(n)$. However, this linear programming approach does not always return the known molecular orientation distribution information when mock spectral information is used to build the instance of the model. The first goal of our study is to figure out the reason that causes the failure of our linear programming model. After that, we also want to know with what spectral information for what test cases can the correct molecular orientation be expected when using linear programming. To achieve these goals, a simplified molecular model is designated to study the nature of our linear programming model. With the information gained, we further apply the linear programming approach to various test cases in order to verify whether it can be systematically applied in different circumstances. We have achieved the following conclusions:with the help of simplified molecular model, lack of sufficient spectral information in the linear programming instances is the reason that the LP solver does not return the target composition. When studying one type of realistic molecular model at surfaces, even combining all three spectral information of IR, Raman and SFG to build the LP instances, it is not sufficient to obtain the target composition for most test cases. In When studying different types of realistic molecular models at surfaces, Raman or SFG spectral information alone is sufficient to obtain the target composition when candidates of each molecular model expanded in $[0^{\circ}$, $90^{\circ})$ on $\theta$. When candidates of each molecular model expanded in $[0^{\circ}$, $180^{\circ}]$ on $\theta$, excluding $90^{\circ}$, SFG spectral information needs to be combined with IR or Raman to obtain the target composition. When the slack variable is introduced to each spectral technique, the case of different types of realistic molecular models at surfaces is considered. When each molecular model's candidates expanded in $[0^{\circ}$, $90^{\circ})$ on $\theta$, Raman spectral information alone is sufficient to obtain the target composition. When each molecular model's candidates expanded in $[0^{\circ}$, $180^{\circ}]$ on $\theta$, excluding $90^{\circ}$, the return compositions, of the LP instances using only Raman spectral information and using Raman and SFG spectral information, are both needed to obtain the target composition.