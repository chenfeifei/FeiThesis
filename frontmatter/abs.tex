\newpage
\TOCadd{Abstract}

\begin{center}
\textbf{ABSTRACT}
\end{center}

%Applying linear programming to vibrational spectra to extract the molecular structure at interfaces is a new approach. Research has been done to explore the possibility of using linear programming. However, this new approach has not been studied systematically. We show the use of linear programming using spectral information data first for a toy model, then for situations with real molecules. Appropriateness for the use of the data from IR, Raman, and SFG spectroscopy techniques for our use case is discussed.

Applying linear programming (LP) to spectroscopy techniques, such as IR, Raman and SFG, is a new approach to extract the molecular orientation information at surfaces. In Hung's previous research, he has shown how applying LP results in the computational gain from $O(n!)$ to $O(n)$. However, this LP approach does not always return the known molecular orientation distribution information when mock spectral information is used to build the instance of the model. The first goal of our study is to figure out the cause for the failed LP instances. After that, we also want to know for different cases with what spectral information, can the correct molecular orientation be expected when using LP. To achieve these goals, a simplified molecular model is designated to study the nature of our LP model. With the information gained, we further apply the LP approach to various test cases in order to verify whether it can be systematically applied to different circumstances. We have achieved the following conclusions: with the help of simplified molecular model, the inability to extract a sufficient data set from the given spectral information to build the LP instances is the reason that the LP solver does not return the target composition. When candidates coming from one same molecule, even combining all three spectral information of IR, Raman and SFG, the data set extracted is still not sufficient in order to obtain the target composition for most cases. When candidates are coming from different molecules, Raman or SFG spectral information alone contains sufficient data set to obtain the target composition when candidates of each molecule expanded in $[0^{\circ}$, $90^{\circ})$ on $\theta$. When candidates of each molecule expanded in $[0^{\circ}$, $180^{\circ}]$ on $\theta$, excluding $90^{\circ}$, SFG spectral information needs to combine with IR or Raman in order to obtain the sufficient data set to obtain the target composition. When the slack variable is introduced to each spectral technique, for the case of candidates coming from different molecules, when candidates expanded in $[0^{\circ}$, $90^{\circ})$ on $\theta$, Raman spectral information carries sufficient data set to obtain the target composition. When candidates expanded in $[0^{\circ}$, $180^{\circ}]$ on $\theta$, excluding $90^{\circ}$, SFG and Raman spectral information together carries sufficient data set in order to obtain the target composition.